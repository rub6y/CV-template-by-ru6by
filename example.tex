\documentclass[a4paper, 10pt]{article}

\usepackage{extsizes}
\usepackage[a4paper, margin=1.5cm]{geometry}
\usepackage{graphicx}
\usepackage{array} % for vertical centering in tabular

\usepackage[utf8]{inputenc}   % allows UTF-8 input (for Polish characters)
\usepackage[T1]{fontenc}      % use T1 font encoding for proper diacritics
\usepackage[polish]{babel}    % Polish hyphenation and language settings

\usepackage{xcolor} % ensure you have this
\definecolor{mygrey}{gray}{0.85} % define your custom grey color

\usepackage{tabularx}

\newcommand{\cvtitle}[7]{
    \noindent
    \begin{tabular}{@{} m{0.10\textwidth} m{0.55\textwidth} m{0.25\textwidth} @{}}
        % --- Logo --- 
        \includegraphics[width=\linewidth]{#1} &
        % --- Name ---
        \vspace{-0.5em}
        \begin{minipage}[c]{\linewidth}
            {\fontsize{25pt}{25pt}\selectfont \textsc{#2}} \\
            {\small #3}\\[-2pt]
            {\small #4}
        \end{minipage} &
        % --- Contact info (aligned to top-right) ---
        \raggedleft
        \vspace{0.75em}
        \small
        #5 $\diamond$\\
        #6 $\diamond$\\
        #7 $\diamond$
    \end{tabular}

    \vspace{4pt}
    \noindent\rule{\textwidth}{0.4pt}
}

\newcommand{\cvsection}[1]{%
  \vspace{1.5em}% space before
  \noindent\textbf{\large $\diamond$ \textsc{#1}}\par
  \vspace{-0.5em}% space after
  \noindent\rule{\textwidth}{0.4pt}\vspace{0.6em}%
}

\newcommand{\cventry}[4]{%
  \noindent
  \begin{tabular*}{\textwidth}{@{\extracolsep{\fill}} l r @{}}
    \textbf{#1} & #2 \\[2pt]
    \textit{#3} & \textit{#4} \\
  \end{tabular*}\vspace{0.5em}
}

\newcommand{\cvtable}[6]{%
    \vspace{0.5em}
    \noindent
    \begin{tabularx}{\textwidth}{@{} X X X @{}}
        \textbf{#1} &
        \textbf{#3} &
        \textbf{#5} \\[2pt]

        \textit{#2} &
        \textit{#4} &
        \textit{#6} \\
    \end{tabularx}
}

\newcommand{\cvsubsection}[1]{%
  \noindent
  \makebox[\textwidth]{%
    \colorbox{mygrey}{%
      \begin{minipage}{0.98\textwidth}
        \centering\textbf{#1 \vphantom{p\^{E}}}
      \end{minipage}%
    }%
  }%
  \par\vspace{0.4em}
}


\begin{document}
    
    \cvtitle
    {images/KNMT.jpg} % Obrazek Obok imienia i nazwiska
    {Jan Kowalski} % Imię i nazwisko
    {Wydział Matematyki i Informatyki,}{Uniwersytet Wrocławski} % Nazwa wydziału i uniwerku
    {(+48) 111-222-333}{imejl@gmail.com}{https://github.com/} % Jakieś dane


    \cvsection{Jakaś sekcja} % Tak można rozpoczynać rożne ogólne sekcje CV

        \cvsubsection{\textbf{Jakaś podsekcja}} % Tak podsekcje w szarym prostokącie 
        
            \cventry % komenda 4 argumentu do tworzenia prostyuch entries
            {Wyróżniony Tekst}{Linijka po prawej}                                         
            {Pod teskt}{Następna linijka po prawej}                       

            \cventry % przykład użycia 
            {Wydział Matematyki i Informatyki, Uniwersytet Wrocławski}{Wrocław, Polska}  
            {Kierunek: Matematyka I stopień, tytuł: licencjat}{2022 - 2025}

        \cvsubsection{\textbf{Inna podsekcja}}

            \cvtable % komenda 6 argumentów do tabelki 2 x 3
            {Wyróżniony tekst I}{komentarz}
            {Wyróżniony tekst II}{komentarz}
            {Wyróżniony tekst III}{komentarz}

        \cvsubsection{\textbf{Przykład}}

            \cvtable
            {Topologia Algebraiczna II}{Ocena: wykł. 5, ćw. 5}
            {Logika R}{Ocena: (w trakcie)}
            {Logika III R}{(nie istnieje)}

\end{document}